%%
	%Created by Zhanbo Hua
	%CUMT Beamer
%%
\documentclass{beamer}
\usepackage{graphicx}
\usepackage{amsmath}
\usepackage{cite} % For citations
\usepackage{hyperref} % For clickable links in the references
\usepackage[absolute,overlay]{textpos}
\usepackage{tikz}
\usepackage{url}
\usepackage[UTF8]{ctex}
\usepackage{tabularx}

%代码设置
\usepackage{fancybox}
\usepackage{xcolor}
\usepackage{times}
\usepackage{listings}

\definecolor{mygreen}{rgb}{0,0.6,0}
\definecolor{mygray}{rgb}{0.5,0.5,0.5}
\definecolor{mymauve}{rgb}{0.58,0,0.82}
\newcommand{\Console}{Console}
\lstset{ %
	backgroundcolor=\color{white},   % choose the background color
	basicstyle=\footnotesize\rmfamily,     % size of fonts used for the code
	columns=fullflexible,
	breaklines=true,                 % automatic line breaking only at whitespace
	captionpos=b,                    % sets the caption-position to bottom
	tabsize=4,
	commentstyle=\color{mygreen},    % comment style
	escapeinside={\%*}{*)},          % if you want to add LaTeX within your code
	keywordstyle=\color{blue},       % keyword style
	stringstyle=\color{mymauve}\ttfamily,     % string literal style
	numbers=left, 
	%frame=single,
	rulesepcolor=\color{red!20!green!20!blue!20},
	%identifierstyle=\color{red},
	language=c
}

\mode<presentation>
{
\usetheme{Madrid}
\usecolortheme{beaver}
}
 
%Change the reference icon to a standard format
\setbeamertemplate{bibliography item}[text]

%------TITLE PAGE------
\title[CUMT]{编码颂国庆 \ 情思寄中秋}
\author[Authors]{华展博}
\institute[CUMT]{计算机科学与技术学院 \\ 中国矿业大学 \\ 08212759@cumt.edu.cn}
\date{2023年10月1日}

\iffalse
%------Highlight the title of the current section------
\AtBeginSection[]
{
	\begin{frame}
		\frametitle{目录}
		\tableofcontents[currentsection]
	\end{frame}
}
\fi

\begin{document}

% Logo at the top center
\begin{textblock*}{5cm}(5.7cm,0.1cm) % Adjust the position as needed
\includegraphics[width=1.5cm]{fig/logo0.png}
\end{textblock*}

%insert title page
\frame{\titlepage}

\iffalse
\makeatletter
\setbeamertemplate{footline}{
  \ifnum\c@framenumber>1%
    \tikz[remember picture,overlay]{
      \node at (current page.north east) [anchor=north east, yshift=-0.2cm] {\includegraphics[width=5cm]{resources/logo1.png}};
    }
  \fi
}
\fi

\makeatother

%insert contents
\begin{frame}
	\frametitle{目录}
	\tableofcontents
\end{frame}

%导言
\section{导言}
	%Frame0
	\begin{frame}{导言}
		\begin{columns}
    		\begin{column}{0.5\textwidth} % Adjust the width as needed
      		\textbf{解字:}
      		\begin{itemize}
        		\item “或,邦也。从囗从戈,以守一。一,地也。
        		\item “囗”的加入,强化了国境线的概念。
      		\end{itemize}
    		\end{column}
    		\begin{column}{0.5\textwidth} % Adjust the width as needed
      		\begin{figure}
        		\includegraphics[width=0.7\linewidth]{fig/guofan.png} % Adjust the path and size of the image
        		\caption{國}
      		\end{figure}
    		\end{column}
  		\end{columns}
	\end{frame}
	%Frame1
	\begin{frame}{导言}
		\begin{columns}
    		\begin{column}{0.5\textwidth} % Adjust the width as needed
      		\textbf{解字:}
      		\begin{itemize}
        		\item 国家实行王道之治。
      		\end{itemize}
    		\end{column}
    		\begin{column}{0.5\textwidth} % Adjust the width as needed
      		\begin{figure}
        		\includegraphics[width=0.7\linewidth]{fig/guojian.png} % Adjust the path and size of the image
        		\caption{国}
      		\end{figure}
    		\end{column}
  		\end{columns}
	\end{frame}
	%Frame2
	\begin{frame}{导言}
		\begin{itemize}
			\item “国”的字形演变,其实也反映了国家概念的多重内涵和不断完善。
			\item 就中国而言,从远古的小国寡民到当今的泱泱大国,尤其是从积贫积弱、四分五裂的半殖民地、半封建国家到建立新中国迈入新征程,其间经历了多少可歌可泣的历史风云和艰辛奋斗!
		\end{itemize}
	\end{frame}
	
%国庆节
\section{国庆节}
	%Frame1
	\begin{frame}{国庆节}
	\framesubtitle{示例代码}
			\lstinputlisting[lastline=20,
							language=c,
							frame=single,
							caption=Simple Code,
							label=python]
							{code/cpp0}
	\end{frame}
	%FrameTable1
	\begin{frame}{国庆节}
	\framesubtitle{示例表格}
		\begin{table}[h]
		\centering
		\caption{实验结果(radio=1)}
			\begin{tabular}{|c|c|c|c|c|}
			\hline
			实验编号 & 精度 & 召回率 & 覆盖率 & 流行度 \\
			\hline
			0 & 19.15\% & 9.2\% & 88.35\% & 6.3121 \\
			1 & 19.01\% & 9.1\% & 87.85\% & 6.3099 \\
			2 & 19.06\% & 9.11\% & 87.71\% & 6.3076 \\
			3 & 18.92\% & 9.09\% & 88.81\% & 6.3017 \\
			4 & 18.9\% & 9.09\% & 88.3\% & 6.3162 \\
			5 & 18.96\% & 9.15\% & 88.24\% & 6.3114 \\
			6 & 18.7\% & 8.99\% & 88.78\% & 6.3055 \\
			7 & 18.69\% & 8.96\% & 88.3\% & 6.2934 \\
			\hline
			\multicolumn{5}{c}{平均结果 (M=8, N=10, 比例=1)} \\
			\hline
 			ratio=1 & 18.92\% & 9.09\% & 88.29\% & 6.3072 \\
			\hline
			\end{tabular}
		\end{table}
	\end{frame}
	
%中秋节
\section{中秋节}
	%Frame1
	\begin{frame}
	\frametitle{中秋节}
	\framesubtitle{示例公式}
		\textbf{目标函数}\\
		偏置部分主要由三个子部分组成,分别是:
		\begin{minipage}[c][0.35\textheight][c]{\linewidth}
			\centering
			\begin{itemize}
  				\item 训练集中所有评分记录的全局平均数$\mu$,表示了训练数据的总体评分情况,对于固定的数据集,它是一个常数。
  				\item 用户偏置$b_i$:独立于物品特征的因素,表示某一特定用户打分习惯。
  				\item 物品偏置$b_j$:特立于用户兴趣的因素,表示某一特定物品得到的打分情况。	
  			\end{itemize}
		\end{minipage}
		\begin{minipage}[c][0.15\textheight][c]{\linewidth}
			\centering
			\textbf{则偏置部分表示为:}\\
			$b_{ij}=\mu+b_i+b_j$\\
			\textbf{加入偏置项的优化函数J(p,q)为:}\\
			$\mathop{\arg\min}\limits_{p_j q_j}\sum_{i,j\in K}(m_{ij}-\mu-b_i-b_j-q_j^Tp_i)^2+\lambda(\Vert p_i \Vert_2^2+\Vert q_j \Vert_2^2+\Vert b_i \Vert_2^2+\Vert b_j \Vert_2^2)$\\
			同理,这个优化目标也可以用梯度下降法求解,方法相同
		\end{minipage}
	\end{frame}
	%Frame2
	\begin{frame}
	\frametitle{中秋节}
		\begin{minipage}[c][0.4\textheight][c]{\linewidth}
			\centering
			\begin{itemize}
  				\item 示例文字1
  				\item 示例文字2
  				\item 示例文字3
			\end{itemize}
		\end{minipage}
		\begin{minipage}[c][0.4\textheight][c]{\linewidth}
			\centering
			\includegraphics[width=0.5\linewidth]{fig/zq}
		\end{minipage}
	\end{frame}
	
%寄语
\section{寄语}
	\begin{frame}{寄语}
		\textbf{ \hspace*{18pt}编码颂国庆,情思寄中秋}
    \end{frame}
    
%参考文献
\section{参考文献}
	\begin{frame}{参考文献}
		\begin{thebibliography}{9}
			\bibitem{XX2012}
			XX编著;XX,XX审校 (2012)。《XXXXXX》。XX:XXXX出版社。
			\bibitem{XX2018}
			XX,XXX,XXX,XXX (2018)。XXXXXXXXXXXX。《XX大学学报(自然科学版)》,第1期,55-60。
		\end{thebibliography}
	\end{frame}
	
\end{document}